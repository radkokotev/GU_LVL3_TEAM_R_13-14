\documentclass[11pt]{article}
\usepackage[dvips]{graphicx}
\usepackage{amsmath}
\usepackage{amsthm}
\usepackage{amsfonts}
\usepackage{pstricks,pstricks-add,pst-math,pst-xkey}
\usepackage[margin=2.5cm]{geometry}
\title{Team Organisation}
\date{Team R\\Members: Ben Turner, Lukas Vascila, Mihail Tachev}
\begin{document}
\maketitle

As a team of relatively inexperienced software engineers, we felt it
would be inappropriate to conduct full scale and total use of the
scrum model in our team. Whilst the scrum model features some
definite positives, it also has negatives, primarily in the scrum
master role. Nobody on our team has the required training for the
scrum master and so, as a group, we decided it would be detrimental
to have a full time, fully realized scrum master. Instead, we plan
to follow the methodology set out by scrum, that is to say, we shall
be following the sprint/lull concept as suggested by both the
lecture and our timetable and also making use of the regular, fairly
fast paced meetings to monitor progress. As we have yet to get
started on the sprint section of the course, it's a little too soon
to say how we're going to manage it, but the suggestion currently is
not to follow the proposed, daily stand-up meetings as it will
certainly break the flow of our work. To expand, we aren't required
to be in the same building every day, 9 til 5, and so for us to meet
up every day would be a significant waste of time (both in travel
time and in planning other events around it). However, it is going
to be necessary to keep in constant contact during that period, so
we intend to make use of the devices and methods as suggested later
in this document.

In line with the scrum methodology, we, as a team, aren't really
capable of assigning or fulfilling full time roles. To avoid this,
we have given one another pseudo-roles. That is to say, a single
person shall be �in charge� of that area, hopefully to have the best
understanding of it, so he can lead the others in the work and
direct work flow most effectively.

With that in mind, we have decided on several pseudo roles. The
chief architect will be leading the design of the code base and will
be most involved with decided on implementation design. In this
case, that refers predominantly to logical structures and
algorithms. Of course, to lighten the load and to reduce risk and
overhead, the whole team will be joining in with this section with
multiple brain storming sessions being a must.

Another team member will be tasked with overseeing the code as a
whole ensuring the correct implementation of the various algorithms
and helping the team work towards the goal. In this scenario he will
also be tasked with ensuring we're not working with the wrong idea
in mind; he, along with others, will be checking to ensure the code
base is semantically correct, so as to reduce risk of failure at a
later date. Agile development focuses more on iteration of the code
base, and as such, this member will have a very important role in
keeping everybody focussed.

Many projects also require a graphical user interface (referred to
as a GUI from this point onwards), a team member will be tasked with
the design of said GUI. To avoid an unnecessary workload, and to
ensure the GUI looks and feels the best it can, the whole team will
have input towards the final design, as well as possible outside
opinion (permission and necessity depending). Most projects  require
a high degree of readability and interactivity, so it will be
paramount that this section of the project is of the highest
quality.

The chances of having a project that is fully within the knowledge
and understanding of a team is low. As such, we consider it highly
important that we understand the project parameters and how best to
implement the solution. With that in mind, a relative expert is a
position we consider very important, who will function as the
�customer� for whom we are designing this project. To attempt to
implement an effective and high quality solution, we think a member
learning about the deepest depths of the subject would be a key part
of a successful and high quality delivery. This member will learn as
much as he can to provide a full and total understanding of every
parameter. Our project cannot succeed without a deep understanding
of the game; this role is critical.

The roles mentioned so far have followed the idea laid out above:
leading the group through that area of the project but not expected
to succeed alone. However, as in many areas of life, there needs to
be a man with brown shoes and a clipboard keeping everyone on track.
For our project, we have elected to have a manager to keep an eye on
everyone's work, to ensure everyone is working at the rate they
should be, and to ensure a timely and complete delivery. Whilst not
in charge of any one role, it will be the managers job to ensure
everything is on time; no missed deadlines and no missed meetings.
The member will chase people for work, hound them for progress and
hopefully direct the project to the best conclusion it can reach.
This role is broader than the other roles, but not as deep, and full
of paper work.

Our suggested roles currently are:\\

\begin{tabular}{|l|l|l|}
  \hline
  % after \\: \hline or \cline{col1-col2} \cline{col3-col4} ...
  \multicolumn{1}{|c}{\textbf{Name}} & \multicolumn{1}{|c}{\textbf{Role}} &
  \multicolumn{1}{|c|}{\textbf{Reasoning}} \\
  \hline
  Chief Architect & Radko & \parbox[t]{10cm}{Radko doesn't have any particular experience
  in this area but is eager to learn and also use some of the key
  concepts, which we learnt in class, in practice.} \\
  \hline
  Graphics Designer & Lukas V. & \parbox[t]{10cm}{Lukas has taken up two years of
  psychology alongside computing science. That combined with experience
  gained with Java swing make him a good candidate for this role.} \\
  \hline
  Field Expert & Mihail & \parbox[t]{10cm}{Keen on board games and is eager to master and
  go into depth with this ancient game} \\
  \hline
  Lead Programmer & Lukas G. & \parbox[t]{10cm}{TODO} \\
  \hline
  Project Manager & Ben & \parbox[t]{10cm}{Ben has experience  both working in and
  leading a project of this kind from his previous University. He is an
  effective communicator and  well organised.}  \\
  \hline
\end{tabular}\\

To ensure our group works effectively, we have several methods of
communication in mind. Our base of communication will be a Facebook
group. This will be checked by every member of the team on a regular
basis to ensure everybody is up to date with the project. We will
use the page for communication that is not time critical. For time
critical communication we have shared contact information (phone
numbers). We are also planning to make use of Skype for face to face
communication during holidays when, due to weather or distance, we
are unable to meet in person. These meetings will be arranged in an
ad hoc manner. We have also set up a Google Calendar to ensure that
members will be reminded of meeting times and of work deadlines. A
Google Drive folder has also been set up to facilitate the easy
sharing of work and a Github repository will be used to share and
monitor code. These are alongside our regular weekly meetings and
our weekly meeting with our supervisor. Speaking directly to others
is extremely important and will provide valuable insight into
working together and being part of a successful team.

A project does not come without risks. As we are all living and
working apart from one another, lack of communication could become
an issue. We share classes so meeting up with one another should be
simple but in the unlikely chance there is an issue, we will first
attempt to contact them on Facebook (to attempt to keep it casual),
followed by University email and telephone when it becomes more
serious. Our worse case scenario is to carry on without that person,
noting the absence and refusal to communicate in our final
submission. If, for whatever reason, a member refuses to work, we
shall first attempt to resolve the issue internal to the group. If
that fails, we shall have no choice but to escalate the issue to a
supervisor so it is officially noted and then we will continue our
work as best we can without that member. In the unlikely case
someone has to leave the group, the rest of the group will be able
to step in and fill the role that has been left, as everyone will be
kept abreast of the whole project and all aspects of its
development.

There is a risk that a member will attempt to delete, or
accidentally delete, one of the repositories and our work along with
it. To attempt to combat that, regular back ups will be made. In the
case of Github, there is an option for recovery available if one
contracts a site admin quickly. Github will also provide version
control for our code which will help prevent unnecessary damage to
the code.

In addition, we have also decided to institute a majority vote
system. If there is a major decision that needs to be made and there
is disagreement over the outcome, the decision will be put to the
vote amongst the team. As a team of five members, this will always
guarantee an outcome and the whole team has agreed to honour this
arrangement. We are also intending to make use of paired programming
when testing to ensure issues are found and corrected quickly and
simply.


\end{document}
