\documentclass[11pt]{article}
\usepackage[dvips]{graphicx}
\usepackage{amsmath}
\usepackage{amsthm}
\usepackage{amsfonts}
\usepackage{pstricks,pstricks-add,pst-math,pst-xkey}
\usepackage[margin=2.5cm]{geometry}
\usepackage{fancyhdr}
\title{Team Organisation}
\author{Team R\\
Ben Turner -- 2092370t@student.gla.ac.uk\\
Lukas Greblikas -- 1107541g@student.gla.ac.uk\\
Lukas Vascila -- 1105348v@student.gla.ac.uk\\
Mihail Tachev -- 1103588t@student.gla.ac.uk\\
Radko Kotev -- 1103815k@student.gla.ac.uk}
\date{Version 3\\ \today}

\pagestyle{fancy}
\lhead{Group exercise 1}
\begin{document}
\maketitle

\begin{abstract}
The aim of this document is to explain what approach we used to 
organise our team. It will describe what we have decided in regards 
to regular team meetings, member distribution among tasks and 
communication options. It will also expand on our choices of version 
control and what actions to take on conflicts within the team or 
other risks.

Additionally, this document will provide some retrospective on how 
our team has performed during this semester and try to discuss any 
possible changes for next semester.  
\end{abstract}


\section{Initial approach}

As a team of relatively inexperienced software engineers, we felt 
it would be inappropriate to conduct full scale and total use of 
the scrum model in our team. Whilst the scrum model features some 
definite positives, it also has negatives, primarily in the scrum 
master role. Nobody on our team has the required training for the 
scrum master and so, as a group, we decided it would be detrimental 
to have a full time, fully realized scrum master. Instead, we plan 
to follow the methodology set out by scrum, that is to say, we shall 
be following the sprint/lull concept as suggested by both the 
lecture and our timetable and also making use of the regular, fairly 
fast paced meetings to monitor progress. 

\section{Team meetings}

So far we have yet to get started on the sprint section of the course, 
it’s a little too soon to say how we’re going to manage it, but the 
suggestion currently is not to follow the proposed, daily stand-up 
meetings as it will certainly break the flow of our work. Instead we 
have decided on a weekly meeting to be held after lectures on every 
Tuesday. This would give us time from 12pm to at least 2pm, which is 
a considerable amount just for discussing our current progress. For 
other occasions, we are all already required to be present in the Boyd 
Orr building from 10am to 3pm almost every day, hence any impromptu 
meeting can be arranged quite easily. In any case, it is going to be 
necessary to keep in constant contact during the period of our course, 
so we intend to make use of the devices and methods as suggested later 
in this document.

\section{Role Assignment}

In line with the scrum methodology, we, as a team, aren’t really 
capable of assigning or fulfilling full time roles. To avoid this, 
we have given one another pseudo-roles. That is to say, a single person 
shall be ”in charge” of that area, hopefully to have the best 
understanding of it, so he can lead the others in the work and direct 
work flow most effectively.
	With that in mind, we have decided on several pseudo roles. The 
chief architect will be leading the design of the code base and will be 
most involved with decided on implementation design. In this case, that 
refers predominantly to logical structures and algorithms. Of course, 
to lighten the load and to reduce risk and overhead, the whole team will 
be joining in with this section with multiple brain storming sessions 
being a must.

Another team member will be tasked with overseeing the code as a whole 
ensuring the correct implementation of the various algorithms and helping 
the team work towards the goal. In this scenario he will also be tasked 
with ensuring we’re not working with the wrong idea in mind; he, along 
with others, will be checking to ensure the code base is semantically 
correct, so as to reduce risk of failure at a later date. Agile 
development focuses more on iteration of the code base, and as such, this 
member will have a very important role in keeping everybody focussed.

Many projects also require a graphical user interface (referred to as a 
GUI from this point onwards), a team member will be tasked with the 
design of said GUI. To avoid an unnecessary workload, and to ensure the 
GUI looks and feels the best it can, the whole team will have input 
towards the final design, as well as possible outside opinion (permission 
and necessity depending). Most projects require a high degree of 
readability and interactivity, so it will be paramount that this section 
of the project is of the highest quality.

The chances of having a project that is fully within the knowledge and 
understanding of a team is low. As such, we consider it highly important 
that we understand the project parameters and how best to implement the 
solution. With that in mind, a relative expert is a position we consider 
very important, who will function as the ”customer” for whom we are 
designing this project. To attempt to implement an effective and high 
quality solution, we think a member learning about the deepest depths of 
the subject would be a key part of a successful and high quality delivery.
This member will learn as much as he can to provide a full and total 
understanding of every parameter. Our project cannot succeed without a 
deep understanding of the game; this role is critical.

The roles mentioned so far have followed the idea laid out above: leading 
the group through that area of the project but not expected to succeed 
alone. However, as in many areas of life, there needs to be a man with 
brown shoes and a clipboard keeping everyone on track. For our project, 
we have elected to have a manager to keep an eye on everyone’s work, to 
ensure everyone is working at the rate they should be, and to ensure a 
timely and complete delivery. Whilst not in charge of any one role, it 
will be the managers job to ensure everything is on time; no missed 
deadlines and no missed meetings. The member will chase people for work, 
hound them for progress and hopefully direct the project to the best 
conclusion it can reach. This role is broader than the other roles, but 
not as deep, and full of paper work.

\section{Breakdown of Role Assignments}
Our suggested roles currently are:\\

\begin{tabular}{|l|l|l|}
  \hline
  % after \\: \hline or \cline{col1-col2} \cline{col3-col4} ...
  \multicolumn{1}{|c}{\textbf{Name}} & \multicolumn{1}{|c}{\textbf{Role}} &
  \multicolumn{1}{|c|}{\textbf{Reasoning}} \\
  \hline
  Chief Architect & Radko & \parbox[t]{9.25cm}{Radko doesn't have any particular experience
  in this area but is eager to learn and also use some of the key
  concepts, which we learnt in class, in practice.} \\
  \hline
  Graphics Designer & Lukas V. & \parbox[t]{9.25cm}{Lukas has taken up two years of
  psychology alongside computing science. That combined with experience
  gained with Java swing make him a good candidate for this role.} \\
  \hline
  Field Expert & Mihail & \parbox[t]{9.25cm}{Keen on board games and is eager to master and
  go into depth with this ancient game} \\
  \hline
  Lead Programmer & Lukas G. & \parbox[t]{9.25cm}{Attention to detail, wide knowledge of
  different programming languages and experience working in industry makes Lukas good
  candidate for this role.} \\
  \hline
  Project Manager & Ben & \parbox[t]{9.25cm}{Ben has experience  both working in and
  leading a project of this kind from his previous University. He is an
  effective communicator and  well organised.}  \\
  \hline
\end{tabular}\\

\section{Methods of Communication}
To ensure our group works effectively, we have several methods of 
communication in mind. Our base of communication will be a Facebook 
group. This will be checked by every member of the team on a regular 
basis to ensure everybody is up to date with the project. We will use 
the page for communication that is not time critical. For time 
critical communication we have shared contact information (phone 
numbers). We are also planning to make use of Skype for face to face 
communication during holidays when, due to weather or distance, we 
are unable to meet in person. These meetings will be arranged in an 
ad hoc manner. We have also set up a Google Calendar to ensure that 
members will be reminded of meeting times and of work deadlines. 

A Google Drive folder has also been set up to facilitate the easy 
sharing of work and a GitHub repository will be used to share and 
monitor code. These are alongside our regular weekly meetings and our 
weekly meeting with our supervisor. Speaking directly to others is 
extremely important and will provide valuable insight into working 
together and being part of a successful team.


\section{Project Risks}

A project does not come without risks. As we are all living and 
working apart from one another lack of communication could become 
an issue. We share classes so meeting up with one another should be 
simple but in the unlikely chance there is an issue, we will first 
attempt to contact them on Facebook (to attempt to keep it casual), 
followed by University email and telephone when it becomes more 
serious. Our worse case scenario is to carry on without that person, 
noting the absence and refusal to communicate in our final submission. 
If, for whatever reason, a member refuses to work, we shall first 
attempt to resolve the issue internal to the group. If that fails, 
we shall have no choice but to escalate the issue to a supervisor so 
it is officially noted and then we will continue our work as best we 
can without that member. In the unlikely case someone has to leave the 
group, the rest of the group will be able to step in and fill the 
role that has been left, as everyone will be kept abreast of the whole 
project and all aspects of its development.

In the event of a new member entering the team, we have decided that 
the best course of action would be to assign a senior member to sit 
down with the newcomer and explain in great detail what is the aim of 
the project and what is our current progress within it. Looking 
through written documentation and Git repositories would be extremely 
useful to getting some one new on track.

There is a risk that a member will attempt to delete, or accidentally 
delete, one of the repositories and our work along with it. To attempt 
to combat that, regular back ups will be made. In the case of Github, 
there is an option for recovery available if one contracts a site admin 
quickly. Github will also provide version control for our code which 
will help prevent unnecessary damage to the code.

In addition, we have also decided to institute a majority vote system. 
If there is a major decision that needs to be made and there is 
disagreement over the outcome, the decision will be put to the vote 
amongst the team. As a team of five members, this will always guarantee 
an outcome and the whole team has agreed to honour this arrangement. 
If all else fails, involving the projcct supervisor will become necessary 
to resolving conflicts within the group. We are also intending to make 
use of paired programming when testing to ensure issues are found and 
corrected quickly and simply.

\section{Retrospective (as of \today)}

After the first semester we as team now feel a considerable amount of 
progression, both in our team projects and in our own skills as 
professional software developers. 

The weekly meetings have proven to be extremely valuable. They offered 
the team a great opportunity every week to discuss current progress on 
the project and allowed for effective work distribution among members. 
Keeping things balanced in such way really allowed us to put enough 
effort both in our team projects and other coursework.

Our chosen methods of communication and version control have also 
proven to be very effective. Our facebook group has been active 
throughout the first semester and all of our documentation is uploaded 
both on the teams trac wiki page and on Google drive, establishing a 
sufficiently safe backup system if necessary. The majority of work has 
been done using GitHub for version control, it easily allows for all 
members to contribute to the projects from home. This has  made 
development process so much more flexible from a time management 
perspective. 

Member assignments to certain areas of the project have not been as 
effective as proposed in the document, but the basic principles of 
our initial plans have been applied. We did not follow the assigned 
roles completely as specified, but there is significant work 
distribution among team members. Also, we have taken great advantage 
of peer programming during the course of this semester. Team members 
are able to fill in for another's weakness, escalating work progress 
across all areas. 

Possible changes for next semester would possibly be deciding on 
Milestones within our projects and make more sensible commits. Up to 
this point most of the team Git commits have been great alterations 
to the project, rather than them being small chunks of code with 
more detailed description of intended changes by the author. 

Milestones would provide good targets within the projects to work 
towards to.  For now our team has only focused on the end goal, with 
little regard to past iterations. Having a more structured project 
requirements distribution among separate milestones would allow for 
better organization of development in the long run. 

\end{document}
